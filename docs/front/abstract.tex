
% -------------------------------------------------------
%  Abstract
% -------------------------------------------------------


\pagestyle{empty}

\شروع{وسطچین}
\مهم{چکیده}
\پایان{وسطچین}
\بدونتورفتگی
نگارش پایاننامه علاوه بر بخش پژوهش و آمادهسازی محتوا،
مستلزم رعایت نکات فنی و نگارشی دقیقی است 
که در تهیهی یک پایاننامهی موفق بسیار کلیدی و مؤثر است.
از آن جایی که بسیاری از نکات فنی مانند قالب کلی صفحات، شکل و اندازهی قلم، 
صفحات عنوان و غیره در تهیهی پایاننامهها یکسان است،
با استفاده از نرمافزار حروفچینی زیتک %\پاورقی{\XeTeX} 
و افزونهی زیپرشین %\پاورقی{XePersian} 
یک قالب استاندارد برای تهیهی پایاننامهها ارائه گردیده است.
این قالب میتواند برای تهیهی پایاننامههای
کارشناسی و کارشناسی ارشد و نیز رسالهی دکتری مورد استفاده قرار گیرد.
این نوشتار به طور مختصر نحوهی استفاده از این قالب را نشان میدهد.

\پرشبلند
\بدونتورفتگی \مهم{کلیدواژهها}: 
پایاننامه، حروفچینی، قالب، زیپرشین
\صفحهجدید
